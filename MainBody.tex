% 文件名:Mainbody.tex
% 文件描述:以 scuthesis 四川大学学位论文文档类为基础的 LaTeX 模版
% 作者:Kevin T. Lee [hello@lidengju.com]
% 修改日期:2020年1月17日

% 设置文档属性
% 参数说明
% professional: 专业学位
% academic: 学术学位
% master: 硕士
% doctor: 博士
% approval: 送审版本,将不生成声明
% color: 红色川大logo
% eqright: 公式右对齐,需要并在公式前加\hfill,仅支持单行公式
\documentclass[professional,master]{./Template/scuthesis2020}
%\newcommand{\supercite}[1]{\textsuperscript{\cite{#1}}}
\begin{document}
% 设置文档信息
\unitid{10610} % 单位代码
\STUnumber{xxxxxxxxxxxxxxx} % 学号
\CoverTitle{论文标题可能很长} %封面标题
\CoverSubTitle{所以你可以把一部分放到副标题里} %可做封面副标题
\title{论文名称} % 论文全称
\ENGtitle{Title of graduate thesis this could be very very long but do not worry it will break the line automatically……} %论文全称英文
\school{计算机学院} % 培养单位
\ENGschool{Computer Science} % 培养单位英文
\author{XXX} % 作者姓名
\ENGauthor{Kevin T. Lee} % 作者英文名
\supervisor{XXX\quad 教授} % 指导教师
\ENGsupervisor{Prof. XXX} % 指导教师英文
\degreeclass{工程硕士} % 学位类别
\ENGdegreeclass{Master of Engineering} % 学位类别英文
\major{计算机技术} % 领域名称
\ENGmajor{Computer Technology} % 领域名称英文
\hasmajor{1} % 若有领域则为1,否则改为0
\defensedate{二〇二〇年二月} % 论文答辩时间
\grantdate{二〇二〇年二月} % 学位授予时间
\accomplishdate{二〇二〇年二月} % 论文完成时间
\statementdate{February, 2020} % 声明时间
\direction{XXX}
\ENGdirection{Direction Name}
\keywords{研究生学位论文;写作指南;参考模板}
\ENGkeywords{graduate thesis;writing guide;reference template}

% 自动制作封面
\maketitle

\makestatement

% \makestatement
% 设置论文正文前的页码、页眉等
\frontmatter\pagenumbering{Roman}\pagestyle{fancy}\makechaptertitlecenter
\makefancyhdr
% 包含摘要
%!TEX root = ../MainBody.tex

% 中英文摘要
\begin{CHSabstract}
    学位论文是研究生科研工作成果的集中体现,是评判学位申请者学术水平、授予其学位的主要依据,是科研领域重要的文献资料。

博士学位论文表明作者在本门学科上掌握了坚实宽广的基础理论和系统深入的专门知识,在科学和专门技术上做出了创造性的成果,并具有独立从事创新科学研究工作或独立承担专门技术开发工作的能力。

硕士学位论文表明作者在本门学科上掌握了坚实的基础理论和系统的专业知识,对所研究课题有新的见解,并具有从事科学研究工作或独立承担专门技术工作的能力。

硕士专业学位论文表明作者在本专业领域掌握了重要的基础理论和系统的专门知识,对相关专业领域问题有新的见解,并具有解决实际问题的能力。

为提高研究生学位论文的撰写质量,促进学位论文在内容和格式上的规范化,根据《学位论文编写规则》(GB/T 7713.1-2006)和《信息与文献 参考文献著录规则》(GB/T 7714-¬2015)等国家有关标准,学院研究生办公室整理并撰写此文,作为指导性规范,供申请学位的研究生参考,以利于学位论文的撰写、收藏、存储、加工、检索和利用。

写作指南主要包括以下内容,分别为内容规范要求,格式规范要求,书写规范要求以及排版印制要求,同时提供了必要的相关附录材料。

本《写作指南与参考模板》将在使用过程中不断完善,如有问题和建议请及时向我们反馈,以便于改进。 谢谢!

\end{CHSabstract}

\begin{ENGabstract}
	\lipsum
\end{ENGabstract}


\makefancyhdr
% 自动制作目录
\maketoc
% 包含缩略词表
%!TEX root = ../test.tex
\chapter{常用缩略词表}

\begin{tabular}{p{7em}p{25em}}
	BF  & Beamforming           \\
	DOA & Direction of Arrivals
\end{tabular}

% 包含符号表
%!TEX root = ../Manual.tex
\chapter{常用符号表}
\emph{例:}\par
\begin{table}[ht]
	\centering
	\begin{tabular*}{\textwidth}{p{0.3\textwidth}p{0.7\textwidth}}
		\TeX    & TeX                           \\
		\LaTeX  & LaTeX                         \\
		\LaTeXe & LaTeX2e                       \\
		\CTeX   & CTeX                          \\
		\AmS    & American Mathematical Society \\
	\end{tabular*}
\end{table}

% 设置论文正文部分的页码、页眉等
\mainmatter\pagenumbering{arabic}\pagestyle{fancy}\makechaptertitlecenter
% 包含第一章、第二章等等
%!TEX root = ../MainBody.tex

% 第一章
\chapter{绪论}% 使用\cite{}命令引用数据库中文献

\section{本文目的}

\section{相关工作}


\chapter{结构内容}
\section{文字要求}
研究生学位论文(thesis or dissertation)应以汉语撰写(外国语言文学专业学位论文可以要求用其它文字撰写)。
来华留学生(全英文项目)可以用英文撰写学位论文,但须有详细中文摘要(不少于6000字),英文摘要300-800英文实词。
\section{结构组成}
\begin{itemize}
\item	中文封面 
\item	英文封面内页
\item	声明
\item	中英文摘要(关键词) 
\item	目录
\item	插图和附表清单(如有)
\item	符号、标志、缩略语等的注释表(如有)
\item	引言(绪论)
\item	正文
\item	参考文献
\item	附录(如有)
\item	攻读学位期间取得的研究成果
\item	致谢
\end{itemize}
\section{内容要求}
学位论文每部分应另页右页开始,各部分内容的要求如下:
\subsection{封面}
\subsection{声明}
\subsection{摘要(关键词)}
\subsection{目录}
\subsection{图和附表清单(如有)}
\subsection{符号、标志、缩略语等的注释表(如有)}
\subsection{正文}
\subsubsection{引言(或绪言)}
\subsubsection{论文主体}
\subsubsection{结论与展望}
\subsection{参考文献}
\subsection{附录(如有)}
\subsection{致谢}
\subsection{攻读学位期间取得的研究成果}
\include{Chapters/Writing}
\include{Chapters/Printing}
\include{Chapters/Conclusion}

% 设置论文正文后的式样
\backmatter\makechaptertitlecenter
% 按国标自动制作参考文献
\begin{reference}
	% 参考文献数据文件为本目录下的ReferenceBase.bib
	\bibliography{ReferenceBase}
\end{reference}
% 包含在读期间科研成果
%!TEX root = ../MainBody.tex

% 作者在读期间科研成果简介
\chapter{攻读学位期间取得的研究成果}
已发表或已录用的学术论文、已出版的专著/译著、已获授权的专利按参考文献格式 列出。未确定正式录用待发表的成果,不能写入此部分内容,成果若为录用待发表状态, 请备注。科研获奖,列出格式为:获奖人(排名情况).项目名称.奖项名称及等级, 发奖机构,获奖时间。与学位论文相关的其它成果参照参考文献格式列出。

全部研究成果连续编号编排。

\textcolor{red}{
    说明:评语送审版本,须保留此页,已取得的研究成果,按照以下内容进行备注:
\\
    “攻读学位期间所取得的研究成果,以第一作者身份(或其他身份)已在《******》 SCI 期刊/核心期刊上正式发表(或被《******》SCI 期刊/核心期刊正式录用待发表), 按照双盲评阅的要求,成果的详细信息未在此匿名评审论文中列出。”
}

% 包含致谢
\makethanks
\end{document}
