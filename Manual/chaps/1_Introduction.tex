%!TEX root = ../Manual.tex
\chapter{前言}
本文档是《四川大学学位论文~\LaTeX~模版》的说明文档。


四川大学学位论文的工作以前由~dahakawang\footnote{\url{https://github.com/dahakawang/scu_thesis_template}}~、~tan\footnote{\url{http://www.codeforge.com/article/382397}}~等人做过。本模版是在参考~Casper Ti. Vector\footnote{\url{CasperVector@gmail.com}}~~\emph{pkuthss}~模版\cite{pkuthss}的基础上,根据《四川大学硕士、博士学位论文格式》\cite{SCUDissertationFormat}完成的。


Legendary L.\footnote{Legendary Leo \url{https://github.com/cuiao}}是本文档的创建者和维护者。

\section{推荐配置}
本模版的使用和正确编译依赖以下几项:
\begin{description}[style=sameline,labelindent=2em,labelwidth=!]
	\item[中文字体] 本模版需要中文字体的支持。
	\item[\TeX~发行版] 一个支持中文的~\TeX~发行版,推荐~\TeX Live\footnote{\url{https://www.tug.org/texlive/}}~,本模版即在~\TeX Live~上构建。
	\item[文本编辑器] 一个好用的文本编辑器有利于你的写作,推荐~Atom\footnote{\url{https://atom.io/}}~,必备插件为~atom-latex\footnote{\url{https://github.com/thomasjo/atom-latex}}~和~language-latex\footnote{\url{https://github.com/area/language-latex}}~。
	\item[PDF~阅读器] 一个轻量级的~PDF~阅读器有利于提升效率,推荐使用~SumatraPDF\footnote{\url{http://www.sumatrapdfreader.org/free-pdf-reader.html}}~与~atom-latex~插件联用。
\end{description}


\section{模版文件}
